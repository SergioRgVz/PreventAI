section{\forlnameref Requisitos de información}
\label{sec:informationReqs}

Se presentan a continuación las \lnameref{sec:entities} que conforman requisitos de información del sistema. Seguidamente, se facilitan las \lnameref{sec:bussinessRules}. Por último, en esta misma sección, se ilustra el \nameref{sec:conceptualModel}.

Para la relación de atributos de las \lnameref{sec:entities} se usará como referencia los tipos descritos en la Tabla \ref{tab:analysisDataTypes}. Asimismo, se incluye si el atributo es único y si es obligatorio.

\captionsetup[table]{singlelinecheck=true}
\begin{table}[ht]
\centering
\begin{tabular}{|c|p{0.8\linewidth}|} 
\hline
\header{Tipo de dato    & Descripción} \\ \hline
\texttt{texto} & Campo alfanumérico. Si el tamaño es acotado, se usa \texttt{texto(\textit{tamaño})}. \\ \hline
\texttt{entero} & Número entero. \\ \hline
\texttt{natural} & Número natural. \\ \hline
\texttt{booleano} & Verdadero o falso. \\ \hline
\texttt{fecha} & Homónimo. \\ \hline
\texttt{hora} & Homónimo. \\ \hline
\texttt{fechahora} & Campos que incluyen los dos anteriores. \\ \hline
\texttt{uri}    & Enlaces o descriptores (por ejemplo, rutas de archivos). \\ \hline
\texttt{binario} & Archivos incrustados. \\ \hline
\texttt{imagen} & Tipo particular de \texttt{binario}, que almacena una imagen. \\ \hline
\texttt{lista<\textit{tipo}>} & Secuencia de un tipo específico. \\ \hline
\texttt{\textit{Requisito}} & Cualquier tipo definido como requisito de información. \\ \hline
\end{tabular}
\caption{Tipos de datos de análisis}
\label{tab:analysisDataTypes}
\end{table}


\begin{shaded}
Existen varias diferencias importantes entre los requisitos de información y las tablas de las bases de datos, que en ocasiones se confunden. Dichas divergencias son, fundamentalmente:
\begin{itemize}
    \item Los requisitos de información son más parecidos a las clases que a las tablas.
    \item No existen ``claves primarias'', ``claves ajenas'' ni ningún otro tipo de elemento característico de los sistemas de gestión de bases de datos relacionales (y mayormente exclusivo de ellos). Por ello, no deben incluirse campos ``Id'' salvo que realmente el cliente haya requerido que haya un campo identificador único, cosa que no suele ocurrir. Dichos artificios aparecerán en todo caso en la fase de \nameref{sec:design}.
    \item Los nombres de las entidades se indican en singular, salvo que cada entrada de dicha entidad pueda representar a más de un elemento (por ejemplo, una entidad \textit{Ruedas} con los atributos \textit{Rueda 1}, \textit{Rueda 2}, etcétera), lo cual es muy poco habitual. Dicho criterio se mantendrá también cuando tales entidades se conviertan en tablas, en su caso.
    \item Deben emplearse nombres claros para los atributos.
    \item El nombre de los atributos no debe contener el nombre de la entidad. Para una entidad \textit{Usuario}, no es correcto llamar a un atributo \textit{CódigoUsuario}, sino \textit{Código}.
    \item Todo atributo candidato a contener un texto dentro de un conjunto acotado de textos se concibe como una relación a una entidad débil. Por ejemplo, un atributo como \textit{País} no debe ser especificado como tipo \code{texto}, ya que su valor forma parte de un conjunto acotado. En su lugar, el tipo es \code{País}, que deberá referenciar a una entidad \textit{País} con (al menos) un atributo \textit{Nombre}.
\end{itemize}
\end{shaded}

\inputContent{analysis/informationReqs/entities.tex}
\inputContent{analysis/informationReqs/bussinessRules.tex}
\inputContent{analysis/informationReqs/conceptualModel.tex}
\inputContent{analysis/informationReqs/traceabilityMatrix.tex}