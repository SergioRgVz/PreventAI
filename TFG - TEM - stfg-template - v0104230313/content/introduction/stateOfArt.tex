\section{Estado del arte}
\label{sec:stateOfArt}

\begin{shaded}
Introducir aquí el estudio, y finalizar con lo siguiente.
\end{shaded}

En la Tabla \ref{tab:stateOfArt} (pág. \pageref{tab:stateOfArt}) se ofrece un cuadro comparativo de características. La comparativa presenta una pequeña parte de la oferta de productos similares, que se ha intentado que sea suficientemente representativa.

\begin{shaded}
El estudio del estado del arte es trabajo de investigación, no de desarrollo, y por esa razón a veces resulta poco motivador para el alumno. No obstante, se requiere para aprender a cuidarse de no hacer trabajo repetido. Hay ocasiones en las que la adquisición de soluciones de terceros (el estado del arte no se circunscribe únicamente a aplicaciones, sino que también puede incluir bibliotecas específicas) es más óptimo que programar funciones que ya existen en el mercado. Por otra parte, se requiere poder justificar al cliente por qué debe apostar por nuestra solución, y no por otra del mercado. En el ámbito del \ac{TFG}, se aconseja el análisis de entre tres y cinco soluciones existentes. La profundidad del análisis y las características recogidas dependen del proyecto a presentar.
\end{shaded}

\subsection*{\program{TeenSafe}} % El asterisco hace que no se numere ni aparezca en el índice.
\label{sec:teemSafe}
\program{TeenSafe} \cite{teensafe2016} es una aplicación de pago que se instala en un dispositivo iPhone y Android, permitiendo en remoto acceder a SMS, iMessages, llamadas realizadas y contactos. También es posible comprobar la ubicación actual del teléfono. En relación a las redes sociales, el sistema permite leer mensajes de \program{Instagram}, \program{WhatsApp} y \program{Kik Messenger}. De igual manera, puede revisarse el historial de búsqueda y de navegación en Internet.

\begin{figure}[ht]
\centering
\includegraphics[scale=.5]{logoTeenSafe.png}
\label{fig:logoTeenSafe}
\end{figure}

\begin{shaded}
Evidentemente esto es un ejemplo. Debe ser sustituido por la primera de las aplicaciones a revisar, y seguido por otras, cada una en su propia sub-sección.
\end{shaded}