\section{Metodología de desarrollo}
\label{sec:methodology}

\begin{shaded}
Descripción detallada del ciclo de vida y la metodología a utilizar. Debe explicarse y referenciarse (bibliografía) lo necesario para justificar la decisión. El motivo es que cualquier evaluador (y cualquier empresario) debería exigir criterios objetivos por los cuales un ingeniero ha tomado una decisión en lugar de otra. Se requiere por tanto justificar las razones por las que alternativas concretas son escogidas. Para ello es necesario demostrar: a) que la opción elegida es beneficiosa por sí misma (por ejemplo, que existe una comunidad profesional que la secunde); y, b) por comparación, que no hay otras opciones que aporten un beneficio sustancialmente mayor para el tipo de proyecto presentado. En todo caso, no se exige ningún ciclo de vida ni metodología concretos. Otros modelos como el ciclo de vida en cascada, basado en prototipos o métodos ágiles son válidos siempre y cuando se justifiquen los puntos \textit{a} y \textit{b} antes referidos.
\end{shaded}