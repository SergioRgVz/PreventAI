\subsection{Relación de tablas}
\label{sec:tablesRelation}

A continuación se muestra la relación de tablas del modelo. La descripción incluye el nombre de la tabla, el/los requisito/s de información que representa y los campos que contiene. Para cada campo se indica el nombre, si es clave primaria (si aparece subrayado), el tipo físico, el campo de referencia en su caso, y si es único y obligatorio.

\begin{shaded}
Se incluye una tabla de ejemplo. Con motivo de establecer una serie de pautas comunes que permita acelerar y facilitar la comprensión del modelo, para la descripción de las tablas se respetarán los siguientes criterios:
\begin{itemize}
    \item Deben eliminarse tildes y espacios, usando la notación Pascal \cite{wiki2012pascal}.
    \item El primer elemento en aparecer será la clave primaria.
    \item Posteriormente, se indican las claves ajenas por orden alfabético.
    \item Por último, se incluye el resto de los campos en agrupación de significado.
\end{itemize}
\end{shaded}

\def \dbtName {Usuario}
\def \dbtIrq {Usuario (Tabla \ref{tab:irqUser}) [pág. \pageref{tab:irqUser}])}
\def \dbtNumFields {6} % Número de campos. Necesario para que "Campos" aparezca centrado
\def \dbtFields { 
    % Filas de primera a penúltima
    % & \code{NombreCampo} & \code{tipoCampo} & referencia (\code{Tabla.Campo} o vacío)
    %                                         & esÚnico (\checkmark o vacío) & esObligatorio (\checkmark o vacío) \\ \cline{2-6}
    % Última fila
    % \dbtTitle  & \code{NombreCampo} & \code{tipoCampo} & referencia (\textit{primaria}, \code{Tabla.Campo} o vacío)
    %            & esÚnico (sí o no) & esObligatorio (sí o no) \\ \hline
            & \underline{\code{Id}} & \code{int32}          & & \checkmark & \checkmark   \\ \cline{2-6}
            & \code{EsSupervisor}   & \code{boolean}        & &            & \checkmark   \\ \cline{2-6}
            & \code{Nombre}         & \code{varchar(64)}    & &            & \checkmark   \\ \cline{2-6}
            & \code{Apellidos}      & \code{varchar(128)}   & &            & \checkmark   \\ \cline{2-6}
            & \code{Usuario}        & \code{varchar(16)}    & & \checkmark & \checkmark   \\ \cline{2-6}
\dbtTitle   & \code{Contrasenna}    & \code{varchar(16)}    & &            & \checkmark   \\ \hline
}
\createDbTableTable{tab:tabUser}