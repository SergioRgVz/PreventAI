\section{Metodología de desarrollo}
\label{sec:methodology}

% \begin{shaded}
% Descripción detallada del ciclo de vida y la metodología a utilizar. Debe explicarse y referenciarse (bibliografía) lo necesario para justificar la decisión. El motivo es que cualquier evaluador (y cualquier empresario) debería exigir criterios objetivos por los cuales un ingeniero ha tomado una decisión en lugar de otra. Se requiere por tanto justificar las razones por las que alternativas concretas son escogidas. Para ello es necesario demostrar: a) que la opción elegida es beneficiosa por sí misma (por ejemplo, que existe una comunidad profesional que la secunde); y, b) por comparación, que no hay otras opciones que aporten un beneficio sustancialmente mayor para el tipo de proyecto presentado. En todo caso, no se exige ningún ciclo de vida ni metodología concretos. Otros modelos como el ciclo de vida en cascada, basado en prototipos o métodos ágiles son válidos siempre y cuando se justifiquen los puntos \textit{a} y \textit{b} antes referidos.
% \end{shaded}

El ciclo de vida utilizado durante la elaboración del proyecto consistirá en una combinación de metodología SCRUM + Kanban llamada ScrumBan ~\cite{Scrumban}. Esta metodología combina las estrategias ágiles de \program{Scrum} y \program{Kanban}. Inicialmente fue desarrollada para facilitar la transición entre Scrum y Kanban, es ideal para equipos que ya estén familiarizados con alguno de estos enfoques. 

Por un lado, Scrum proporciona una estructura iterativa y tiempo limitado para las tareas, dividido en "sprints" que permiten la planificación y revisión frecuente del progreso. Esto ayuda a mantener un enfoque claro y a adaptarse rápidamente a los cambios. Al final de cada sprint, se revisa el trabajo y se planifica el próximo, lo cual es ideal para mantener la dirección y el enfoque en proyectos a corto plazo.

Kanban, por otro lado, complementa a Scrum con su enfoque en la visualización del flujo de trabajo y la mejora continua. Utiliza un tablero Kanban para mostrar tareas en varias etapas de desarrollo (por ejemplo, "ToDo", "Doing", "Done" y "Verified"), lo que facilita la gestión y priorización de tareas. Esta visibilidad constante del progreso y la capacidad de ajustar rápidamente las prioridades son particularmente útiles en entornos de trabajo dinámicos.

Scrum aporta una estructura iterativa con sprints y revisiones, mientras que Kanban ofrece una gestión visual del trabajo y límites claros en la cantidad de tareas en curso. ~\cite{guzmanScrumbanmetodologia}
Esta metodología es particularmente útil para equipos que busquen una mayor autonomía en la gestión de tareas. Scrumban permite una planificación y adaptación flexibles del flujo de trabajo. 
